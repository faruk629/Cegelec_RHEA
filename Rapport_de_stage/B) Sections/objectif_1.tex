\section{Identification du type de traverse de voie}

Cette section se concentre sur la présentation de la méthode existante, ainsi que sur l'exploration et l'analyse de nouveaux modèles afin de déterminer le type de traverse. Différentes modèles d'IA sont examinés afin de les évaluer et sélectionner le plus performant. 

\subsection{Exploration des Données}
% Dans un premier temps, la collecte des données se concentre sur trois types spécifiques de traverses, à savoir BB1, MB5 et WS. Ces données sont recueillies en fonction du type de segment identifié à l'aide de la méthode mathématique décrite dans la Section 5, et elles sont confirmées visuellement à travers l'interface Hyperion. Ces segment contiennent un certain nombre de profils par type. Le tableau X répertorie le nombre de profils par type qui seront utilisés lors d'une première approche.


%% Dire que j'ai eté sur hyperion pour prendre les données. Que j'ai checké chaque données à l'aide de QGIS puis j'ai vérifier si c'est vraiment ce type de traverse. J'ai aussi reagrder sur ballast FX pour comparer ce ballast si c'est bien ce type. 

Dans cette section, nous aborderons la récupération des données, leur traitement initial ainsi que l'application de techniques de data augmentation pour enrichir notre ensemble de données.
\subsection{Modèle existant} 
Dans cette section, la première approche pour détecter le type de traverse contenu dans un segment est examinée. Cette méthode, déjà existante, repose sur une approche mathématique qui est assez performante, mais qui est sujette à des limitations dans des conditions spécifiques (\textit{cf.} Section 5.2.1). L'identification de la traverse se base sur les profils qui se trouvent dans la Zone 2 (\textit{cf.} Section 2.2). \\

\noindent Cette approche, développée en Java, est décomposée en 3 étapes. Dans la première étape (cf. Section 5.1.1), les profils présents dans un segment sont filtrés pour ne garder que ceux contenant des traverses. Ensuite (\textit{cf.} Section 5.1.2), ces profils sont analysés pour identifier le type spécifique de traverse. Enfin (\textit{cf.} Section 5.1.3), un tableau recense toutes les traverses identifiées avec leur fréquence d'apparition, ce tableau étant utilisé comme score pour déterminer le type du segment.


\subsubsection{Tri des profils}


L'analyse du type de traverse débute par un filtrage des profils contenus dans un segment afin de ne conserver que ceux qui contiennent des traverses. Pour ce faire, on se base sur deux zones plates qui sont communes à toutes les traverses, comme illustré par la Figure 14.  

\begin{figure}[H]
            \centering

           \fbox{ \includegraphics[width=7.5cm]{images/sleepers_flat.png}}   
            \caption{Partie plate commune à toutes les traverses - Cegelec \cite{RHEA}} 
        \end{figure}

\noindent Ensuite, des fonctions polynomiales du premier degré sont appliquées sur ces zones afin de comparer l'écart-type de la régression avec un seuil prédéfini de 2,5. Si l'écart-type dans l'une des deux zones est inférieur à ce seuil, le profil est considéré comme celui d'une traverse (\textit{cf.} Figure 15a) ; sinon, il est classé comme étant du ballast (\textit{cf.} Figure 15b).
 
 \begin{figure}[H]
      \centering
      \fbox{\subcaptionbox{Profil contenant une traverse}{\includegraphics[width=6cm]{images/sleepers_detection_yes.png}}}
      \qquad % Add some space between the two subfigures
      \fbox{\subcaptionbox{Profil contenant du ballast}{\includegraphics[width=5.8cm]{images/sleepers_detection_no.png}}}

      \caption{Identification du profil contenant une traverse versus ballast}

    \end{figure}
        

\subsubsection{Identification du type de traverse}
Pour identifier le type de traverse, une méthode statique est utilisée, s'appuyant sur une bibliothèque de profils de référence disponibles dans un fichier JSON. Ces profils théoriques sont dérivés des plans compilés et définissent la forme à suivre pour chaque type de traverse. \\

\noindent Le processus d'identification se déroule en plusieurs étapes.
\begin{itemize}
\item Tout d'abord, une hauteur de référence est déterminée en fonction de la configuration spécifique de la traverse. Les emplacements dans la zone x, définis dans la bibliothèque, indiquent où mesurer cette hauteur de référence. Le choix stratégique de la zone x vise à générer des écarts en fonction du type de traverse.

\item Par la suite, le profil mesuré et le profil théorique sont comparés en les superposant sur leurs points de hauteur de référence, et l'écart entre les deux est déterminé. Ce dernier est quantifié en calculant l'intégrale des différences entre les deux profils (\textit{cf.} Figure 16).

\item Enfin, un score est calculé pour chaque type de traverse en fonction de la surface d'écart obtenue lors de la comparaison. Un score plus faible indique une meilleure correspondance entre le profil mesuré et le modèle théorique.

\end{itemize}
\begin{figure}[H]
            \centering

           \fbox{ \includegraphics[width=12cm]{images/identification_sleepers.png}}   
            \caption{Identification de la traverse - Cegelec \cite{RHEA}} 
        \end{figure}


\subsubsection{Identification du type de segment}
L'identification du type de segment repose sur l'analyse d'un tableau qui répertorie, pour chaque type de traverse, le nombre de profils qui lui sont associés ainsi que la moyenne des scores de correspondance de ces profils (\textit{cf.} Figure 17). Le type de traverse attribué au segment est celui dont les profils associés ont le meilleur score. \\

\begin{figure}[H]
            \centering

           \fbox{ \includegraphics[width=10cm]{images/segment_identification.png}}   
            \caption{Identification du type de segment - Cegelec \cite{RHEA}} 
        \end{figure}

\noindent Cependant, des règles de vote sont appliquées lors de l'analyse des profils de segments. Si tous les profils sont jugés de mauvaise qualité, le segment est rejeté. De même, si la majorité des profils ont un score de correspondance supérieur à 20, le segment est également rejeté. Ces règles garantissent la fiabilité de l'identification du type de segment.


\subsubsection{Limites du modèle}
% % Discutez des limites ou des lacunes du modèle mathématique existant.
% % Identifiez les aspects pour lesquels l'intelligence artificielle peut apporter des améliorations.Actuellement, la première approche développée en Java pour détecter le type de traverse repose sur une méthode mathématique qui fonctionne de manière raisonnable. Toutefois, certaines limitations se manifestent dans des conditions spécifiques, telles qu'une quantité accrue de ballast, la présence de rouille sur les traverses, des buissons ou une végétation \\

% En résumé, cette méthode d'identification du type de traverse dans un segment propose une approche bien structurée en trois étapes distinctes. D'abord, elle effectue un filtrage des profils pour ne conserver que ceux qui contiennent des traverses, puis elle procède à une analyse approfondie pour déterminer le type spécifique de traverse. Enfin, elle se sert d'un tableau répertoriant les types de traverses identifiés avec leurs fréquences d'apparition et leurs scores pour conclure sur le type du segment.\\

% \noindent Cependant, 
Bien que cette méthode soit performante dans de nombreuses situations, elle présente des lacunes dans des conditions où des facteurs perturbateurs tels que la végétation ou la rouille peuvent affecter la précision de la détection. De plus, certains types de traverses moins courants sur le terrain, ainsi que les segments contenant des composants spécifiques tels que des crocodiles ou des balises ETCS, ne sont pas pris en compte dans le processus d'identification. \\

\noindent Malgré ces limitations, cette méthode reste efficace dans de nombreuses situations. Néanmoins, les variations subtiles peuvent représenter des défis significatifs, mettant en évidence la nécessité d'explorer des approches alternatives. Par exemple, l'utilisation de techniques basées sur l'intelligence artificielle pourrait offrir des solutions plus robustes et adaptatives.


\subsection{Approche 1}

Cette approche consiste à utiliser les données préparées dans la section 5.1  afin de développer un système de classification en utilisant un réseau de neurones convolutionnel (CNN) avec une convolution à une dimension. Dans cette approche, seuls les points sur l'axe vertical (y) de la traverse sont pris en compte pour l'analyse.


\subsubsection{Modèles et Analyse}
Cette section présente une gamme de modèles utilisés pour l'identification du type de traverses, accompagnée d'une analyse comparative visant à déterminer le meilleur modèle pour l'identification du type de traverses. \\

\subsection{Approche 2}
Cette approche consiste à utiliser à la fois les coordonnées horizontales (x) et verticales (y) des points de la traverse pour développer un système de classification. En intégrant ces deux dimensions, nous cherchons à obtenir une représentation plus complète des caractéristiques des traverses, ce qui pourrait améliorer la précision de notre modèle de classification par rapport à l'approche précédente.

\subsubsection{Modèles et Analyse}
Cette section présente une gamme de modèles utilisés pour l'identification du type de traverses, accompagnée d'une analyse comparative visant à déterminer le meilleur modèle pour l'identification du type de traverses. \\

\subsection{Approche 3}
Cette approche  consiste à utiliser les coordonnées horizontales (x) et verticales (y) des points de la traverse, ainsi que l'intensité des données. En intégrant ces trois dimensions, nous cherchons à obtenir une représentation encore plus complète des caractéristiques des traverses. Cela nous permettra d'explorer de manière plus approfondie les variations et les motifs présents dans les données, avec pour objectif d'améliorer la précision et la robustesse de notre système de classification.
\subsubsection{Modèles et Analyse}
Cette section présente une gamme de modèles utilisés pour l'identification du type de traverses, accompagnée d'une analyse comparative visant à déterminer le meilleur modèle pour l'identification du type de traverses. \\

%  \noindent Les modèles choisis seront entraînés à l'aide de la validation croisée sur des données d'entraînement composé de 2000 profils et évalués sur des données de test de 400 profils, afin de mesurer les performances de généralisation des modèles. \\

% \noindent En outre, chaque modèle sera évalué en utilisant différentes configurations d'hyperparamètres, comprenant des tailles de batchs de 8, 16, 32 et 64, ainsi que des taux d'apprentissage de 0.01, 0.001 et 0.0001. L'objectif est de sélectionner, pour chaque modèle, la configuration offrant les meilleures performances pour l'analyse comparative. \\

% \noindent L'analyse débutera avec des modèles simples pour ensuite être plus sophistiquées. \\

% \paragraph{Modèle 1} \\
% Ce modèle est composé de 3 couches et les résultats obtenues par les validation crosiées sont disponible dans les annexes avec la référence A. Ce modèle est composé de x paramètre entrainable et y paramètre non entrainable.
% \begin{figure}[H]
  \centering
  \begin{tikzpicture}
    \node[input,minimum width=2cm, minimum height=2cm] (x) at (-0.50,0)
    {\small Input};
 
    \node[conv,rotate=90,minimum width=4.5cm] (conv1) at (1.25,0) 
    {\small\textbf{Conv1D (32,2) + ReLU}};
    
    \node[flat,rotate=90,minimum width=4.5cm] (flat1) at (2.5,0) {\small\textbf{Flatten}};
    
    \node[dense,rotate=90,minimum width=4.5cm] (dense1) at (3.75,0) {\small\textbf{Dense (3) + Softmax}};

    \node[output,minimum width=2cm, minimum height=2cm] (y) at (5.5,0) 
    {\small\textbf{$Output$}};


    \draw[-] (x) -- (conv1);
    \draw[-] (conv1) -- (flat1);
    \draw[-] (flat1) -- (dense1);
    \draw[-] (dense1) -- (y);

  \end{tikzpicture}
  \vskip 6px
  \caption{An illustration of.}
\end{figure}

% \paragraph{Modèle 2}

% \input{D) Models/model2}

% \paragraph{Modèle 3}

% \begin{figure}[H]
  \centering
  \begin{tikzpicture}
    \node[input,minimum width=2cm, minimum height=2cm] (x) at (-0.50,0)
    {\small Input};
 
    \node[conv,rotate=90,minimum width=4.5cm] (conv1) at (1.25,0) 
    {\small\textbf{Conv1D (32,2) + ReLU}};
    
    \node[pool,rotate=90,minimum width=4.5cm] (pool1) at (2.5,0) {\small\textbf{MaxPooling1D (2)}};

    \node[drop,rotate=90,minimum width=4.5cm] (drop1) at (3.75,0) {\small\textbf{Dropout (0.5)}};

    \node[conv,rotate=90,minimum width=4.5cm] (conv2) at (5,0) 
    {\small\textbf{Conv1D (64,3) + ReLU}};

    \node[pool,rotate=90,minimum width=4.5cm] (pool2) at (6.25,0) {\small\textbf{MaxPooling1D (2)}};
    
    \node[flat,rotate=90,minimum width=4.5cm] (flat1) at (7.5,0) {\small\textbf{Flatten}};
    
    \node[dense,rotate=90,minimum width=4.5cm] (dense1) at (8.75,0) {\small\textbf{Dense (128) + ReLU}};
    
    \node[drop,rotate=90,minimum width=4.5cm] (drop2) at (10,0) {\small\textbf{Dropout (0.5)}};
    
    \node[dense,rotate=90,minimum width=4.5cm] (dense2) at (11.25,0) {\small\textbf{Dense (3) + Softmax}};

    \node[output,minimum width=2cm, minimum height=2cm] (y) at (13,0) 
    {\small\textbf{$Output$}};


    \draw[-] (x) -- (conv1);
    \draw[-] (conv1) -- (pool1);
    \draw[-] (pool1) -- (drop1);
    \draw[-] (drop1) -- (conv2);
    \draw[-] (conv2) -- (pool2);
    \draw[-] (pool2) -- (flat1);
    \draw[-] (flat1) -- (dense1);
    \draw[-] (dense1) -- (drop2);
    \draw[-] (drop2) -- (dense2);
    \draw[-] (dense2) -- (y);
  \end{tikzpicture}
  \vskip 6px
  \caption{An illustration of.}
\end{figure}





% Détaillez le processus d'intégration de l'IA dans le modèle existant.
% Expliquez comment les données sont utilisées, comment le modèle est formé, etc.
\subsection{Comparaison des Résultats}

% In this phase, a structure for a 1D-CNN model was proposed as a baseline. Then, to investigate the
% impact of different structures on the performance of the proposed model, another five CNN models
% with different structures in terms of dropout layer exclusion, kernel size, filter size, the inclusion of an
% additional convolutional layer, and type of convolutional and max pooling layers were constructed.
% After that, four machine learning models were developed to measure the efficiency of our proposed
% model compared to the established machine learning models.



%  In this paper, we aim to address such
% issues in predicting software defects. We propose a novel structure of 1-
% Dimensional Convolutional Neural Network (1D-CNN), a deep learning
% architecture to extract useful knowledge, identifying and modelling the knowledge in the data sequence, reduce overfitting, and finally, predict whether the
% units of code are defects prone. We design large-scale empirical studies to
% reveal the proposed model’s effectiveness by comparing four established traditional machine learning baseline models and four state-of-the-art baselines
% in software defect prediction based on the NASA datasets. The experimental
% results demonstrate that in terms of f-measure, an optimal and modest 1DCNN with a dropout layer outperforms baseline and state-of-the-art models
% by 66.79% and 23.88%, respectively, in ways that minimize overfitting and
% improving prediction performance for software defects. According to the
% results, 1D-CNN seems to be successful in predicting software defects and may
% be applied and adopted for a practical problem in software engineering. This,
% in turn, could lead to saving software development resources and producing
% more reliable software
\noindent Chaque modèle, avec sa configuration la plus performante, sera ensuite évalué en fonction des critères et pondérations. (Veille Technologique)
% Dire quelle fonction de cout on a utilsier. Et pourquoi. 

% Présentez la méthodologie de validation des résultats obtenus à partir du modèle mathématique et de l'IA.
% Comparez les performances des deux approches, en soulignant les avantages de l'utilisation de l'IA.
\subsection{Discussion sur les Avantages et les Limitations}

Discussion des avantages et limitations que l'intelligence artificielle apporte par rapport au modèle mathématique seul.
% Identifiez les limitations possibles de l'IA dans ce contexte.
\subsection{Perspectives Futures}
Suggestion des amélioration possibles.
% Suggérez des améliorations possibles de l'intégration de l'IA dans le modèle.
% Identifiez des domaines de recherche ou d'exploration potentiels.
% Conclusion de la Section :

% Résumez les principales contributions de l'utilisation de l'IA dans le contexte du modèle mathématique existant.
% Préparez la transition vers les sections suivantes de votre mémoire.
\subsection{Conclusion}