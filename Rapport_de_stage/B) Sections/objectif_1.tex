\section{Identification du type de traverse de voie}

Cette section se concentre sur la présentation de la méthode existante, ainsi que sur l'exploration et l'analyse de nouveaux modèles afin de déterminer le type de traverse. Différentes modèles d'IA sont examinés afin de les évaluer et sélectionner le plus performant. 

\subsection{Exploration des Données}
% Dans un premier temps, la collecte des données se concentre sur trois types spécifiques de traverses, à savoir BB1, MB5 et WS. Ces données sont recueillies en fonction du type de segment identifié à l'aide de la méthode mathématique décrite dans la Section 5, et elles sont confirmées visuellement à travers l'interface Hyperion. Ces segment contiennent un certain nombre de profils par type. Le tableau X répertorie le nombre de profils par type qui seront utilisés lors d'une première approche.


%% Dire que j'ai eté sur hyperion pour prendre les données. Que j'ai checké chaque données à l'aide de QGIS puis j'ai vérifier si c'est vraiment ce type de traverse. J'ai aussi reagrder sur ballast FX pour comparer ce ballast si c'est bien ce type. 

Dans cette section, nous aborderons la récupération des données, leur traitement initial ainsi que l'application de techniques de data augmentation pour enrichir notre ensemble de données.
\input{B) Sections/Objectif_1/Modèle Existant}
\subsection{Approche 1}

Cette approche consiste à utiliser les données préparées dans la section 5.1  afin de développer un système de classification en utilisant un réseau de neurones convolutionnel (CNN) avec une convolution à une dimension. Dans cette approche, seuls les points sur l'axe vertical (y) de la traverse sont pris en compte pour l'analyse.


\subsubsection{Modèles et Analyse}
Cette section présente une gamme de modèles utilisés pour l'identification du type de traverses, accompagnée d'une analyse comparative visant à déterminer le meilleur modèle pour l'identification du type de traverses. \\

\subsection{Approche 2}
Cette approche consiste à utiliser à la fois les coordonnées horizontales (x) et verticales (y) des points de la traverse pour développer un système de classification. En intégrant ces deux dimensions, nous cherchons à obtenir une représentation plus complète des caractéristiques des traverses, ce qui pourrait améliorer la précision de notre modèle de classification par rapport à l'approche précédente.

\subsubsection{Modèles et Analyse}
Cette section présente une gamme de modèles utilisés pour l'identification du type de traverses, accompagnée d'une analyse comparative visant à déterminer le meilleur modèle pour l'identification du type de traverses. \\

\subsection{Approche 3}
Cette approche  consiste à utiliser les coordonnées horizontales (x) et verticales (y) des points de la traverse, ainsi que l'intensité des données. En intégrant ces trois dimensions, nous cherchons à obtenir une représentation encore plus complète des caractéristiques des traverses. Cela nous permettra d'explorer de manière plus approfondie les variations et les motifs présents dans les données, avec pour objectif d'améliorer la précision et la robustesse de notre système de classification.
\subsubsection{Modèles et Analyse}
Cette section présente une gamme de modèles utilisés pour l'identification du type de traverses, accompagnée d'une analyse comparative visant à déterminer le meilleur modèle pour l'identification du type de traverses. \\

%  \noindent Les modèles choisis seront entraînés à l'aide de la validation croisée sur des données d'entraînement composé de 2000 profils et évalués sur des données de test de 400 profils, afin de mesurer les performances de généralisation des modèles. \\

% \noindent En outre, chaque modèle sera évalué en utilisant différentes configurations d'hyperparamètres, comprenant des tailles de batchs de 8, 16, 32 et 64, ainsi que des taux d'apprentissage de 0.01, 0.001 et 0.0001. L'objectif est de sélectionner, pour chaque modèle, la configuration offrant les meilleures performances pour l'analyse comparative. \\

% \noindent L'analyse débutera avec des modèles simples pour ensuite être plus sophistiquées. \\

% \paragraph{Modèle 1} \\
% Ce modèle est composé de 3 couches et les résultats obtenues par les validation crosiées sont disponible dans les annexes avec la référence A. Ce modèle est composé de x paramètre entrainable et y paramètre non entrainable.
% \begin{figure}[H]
  \centering
  \begin{tikzpicture}
    \node[input,minimum width=2cm, minimum height=2cm] (x) at (-0.50,0)
    {\small Input};
 
    \node[conv,rotate=90,minimum width=4.5cm] (conv1) at (1.25,0) 
    {\small\textbf{Conv1D (32,2) + ReLU}};
    
    \node[flat,rotate=90,minimum width=4.5cm] (flat1) at (2.5,0) {\small\textbf{Flatten}};
    
    \node[dense,rotate=90,minimum width=4.5cm] (dense1) at (3.75,0) {\small\textbf{Dense (3) + Softmax}};

    \node[output,minimum width=2cm, minimum height=2cm] (y) at (5.5,0) 
    {\small\textbf{$Output$}};


    \draw[-] (x) -- (conv1);
    \draw[-] (conv1) -- (flat1);
    \draw[-] (flat1) -- (dense1);
    \draw[-] (dense1) -- (y);

  \end{tikzpicture}
  \vskip 6px
  \caption{An illustration of.}
\end{figure}

% \paragraph{Modèle 2}

% \input{D) Models/model2}

% \paragraph{Modèle 3}

% \begin{figure}[H]
  \centering
  \begin{tikzpicture}
    \node[input,minimum width=2cm, minimum height=2cm] (x) at (-0.50,0)
    {\small Input};
 
    \node[conv,rotate=90,minimum width=4.5cm] (conv1) at (1.25,0) 
    {\small\textbf{Conv1D (32,2) + ReLU}};
    
    \node[pool,rotate=90,minimum width=4.5cm] (pool1) at (2.5,0) {\small\textbf{MaxPooling1D (2)}};

    \node[drop,rotate=90,minimum width=4.5cm] (drop1) at (3.75,0) {\small\textbf{Dropout (0.5)}};

    \node[conv,rotate=90,minimum width=4.5cm] (conv2) at (5,0) 
    {\small\textbf{Conv1D (64,3) + ReLU}};

    \node[pool,rotate=90,minimum width=4.5cm] (pool2) at (6.25,0) {\small\textbf{MaxPooling1D (2)}};
    
    \node[flat,rotate=90,minimum width=4.5cm] (flat1) at (7.5,0) {\small\textbf{Flatten}};
    
    \node[dense,rotate=90,minimum width=4.5cm] (dense1) at (8.75,0) {\small\textbf{Dense (128) + ReLU}};
    
    \node[drop,rotate=90,minimum width=4.5cm] (drop2) at (10,0) {\small\textbf{Dropout (0.5)}};
    
    \node[dense,rotate=90,minimum width=4.5cm] (dense2) at (11.25,0) {\small\textbf{Dense (3) + Softmax}};

    \node[output,minimum width=2cm, minimum height=2cm] (y) at (13,0) 
    {\small\textbf{$Output$}};


    \draw[-] (x) -- (conv1);
    \draw[-] (conv1) -- (pool1);
    \draw[-] (pool1) -- (drop1);
    \draw[-] (drop1) -- (conv2);
    \draw[-] (conv2) -- (pool2);
    \draw[-] (pool2) -- (flat1);
    \draw[-] (flat1) -- (dense1);
    \draw[-] (dense1) -- (drop2);
    \draw[-] (drop2) -- (dense2);
    \draw[-] (dense2) -- (y);
  \end{tikzpicture}
  \vskip 6px
  \caption{An illustration of.}
\end{figure}





% Détaillez le processus d'intégration de l'IA dans le modèle existant.
% Expliquez comment les données sont utilisées, comment le modèle est formé, etc.
\subsection{Comparaison des Résultats}

% In this phase, a structure for a 1D-CNN model was proposed as a baseline. Then, to investigate the
% impact of different structures on the performance of the proposed model, another five CNN models
% with different structures in terms of dropout layer exclusion, kernel size, filter size, the inclusion of an
% additional convolutional layer, and type of convolutional and max pooling layers were constructed.
% After that, four machine learning models were developed to measure the efficiency of our proposed
% model compared to the established machine learning models.



%  In this paper, we aim to address such
% issues in predicting software defects. We propose a novel structure of 1-
% Dimensional Convolutional Neural Network (1D-CNN), a deep learning
% architecture to extract useful knowledge, identifying and modelling the knowledge in the data sequence, reduce overfitting, and finally, predict whether the
% units of code are defects prone. We design large-scale empirical studies to
% reveal the proposed model’s effectiveness by comparing four established traditional machine learning baseline models and four state-of-the-art baselines
% in software defect prediction based on the NASA datasets. The experimental
% results demonstrate that in terms of f-measure, an optimal and modest 1DCNN with a dropout layer outperforms baseline and state-of-the-art models
% by 66.79% and 23.88%, respectively, in ways that minimize overfitting and
% improving prediction performance for software defects. According to the
% results, 1D-CNN seems to be successful in predicting software defects and may
% be applied and adopted for a practical problem in software engineering. This,
% in turn, could lead to saving software development resources and producing
% more reliable software
\noindent Chaque modèle, avec sa configuration la plus performante, sera ensuite évalué en fonction des critères et pondérations. (Veille Technologique)
% Dire quelle fonction de cout on a utilsier. Et pourquoi. 

% Présentez la méthodologie de validation des résultats obtenus à partir du modèle mathématique et de l'IA.
% Comparez les performances des deux approches, en soulignant les avantages de l'utilisation de l'IA.
\subsection{Discussion sur les Avantages et les Limitations}

Discussion des avantages et limitations que l'intelligence artificielle apporte par rapport au modèle mathématique seul.
% Identifiez les limitations possibles de l'IA dans ce contexte.
\subsection{Perspectives Futures}
Suggestion des amélioration possibles.
% Suggérez des améliorations possibles de l'intégration de l'IA dans le modèle.
% Identifiez des domaines de recherche ou d'exploration potentiels.
% Conclusion de la Section :

% Résumez les principales contributions de l'utilisation de l'IA dans le contexte du modèle mathématique existant.
% Préparez la transition vers les sections suivantes de votre mémoire.
\subsection{Conclusion}