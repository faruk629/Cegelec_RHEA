\subsection{Exploration des Données}
% Dans un premier temps, la collecte des données se concentre sur trois types spécifiques de traverses, à savoir BB1, MB5 et WS. Ces données sont recueillies en fonction du type de segment identifié à l'aide de la méthode mathématique décrite dans la Section 5, et elles sont confirmées visuellement à travers l'interface Hyperion. Ces segment contiennent un certain nombre de profils par type. Le tableau X répertorie le nombre de profils par type qui seront utilisés lors d'une première approche.


%% Dire que j'ai eté sur hyperion pour prendre les données. Que j'ai checké chaque données à l'aide de QGIS puis j'ai vérifier si c'est vraiment ce type de traverse. J'ai aussi reagrder sur ballast FX pour comparer ce ballast si c'est bien ce type. 

Dans cette section, nous aborderons la récupération des données, leur traitement initial ainsi que l'application de techniques de data augmentation pour enrichir notre ensemble de données.